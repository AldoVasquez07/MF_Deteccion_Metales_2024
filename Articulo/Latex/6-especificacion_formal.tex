\section{Especificación Formal}

\subsection{Clase Metal}

La clase \textbf{Metal} representa un metal detectado en una muestra de agua. Los atributos principales son:

\begin{itemize}
    \item \texttt{id}: Un identificador único para el metal.
    \item \texttt{nombre}: El nombre del metal, representado como una secuencia de caracteres.
    \item \texttt{concentracion}: La concentración del metal en la muestra, que debe ser un número real positivo.
\end{itemize}

Además, la clase contiene dos invariantes:

\begin{itemize}
    \item La longitud del nombre debe ser mayor que cero.
    \item La concentración debe ser mayor que cero.
\end{itemize}

El código en VDM++ de la clase es el siguiente:

\begin{lstlisting}
class Metal

types
public String = seq of char

instance variables
private id: nat1;
private nombre: String;
private concentracion: real;

inv len nombre > 0;
inv concentracion > 0;

operations
public Metal: nat1 * String * real ==> Metal
Metal(i, nom, con) == ( id := i; nombre := nom;
	concentracion := con; );

public getId: () ==> nat1
getId() == ( return id; );

public getConcentracion: () ==> real
getConcentracion() == ( return concentracion; );

end Metal
\end{lstlisting}

\subsection{Clase Muestra}

La clase \textbf{Muestra} modela una muestra de agua que contiene diferentes metales. Los atributos principales de esta clase son:

\begin{itemize}
    \item \texttt{id}: Un identificador único de la muestra.
    \item \texttt{ubicacion}: La ubicación donde se tomó la muestra.
    \item \texttt{fecha} y \texttt{hora}: La fecha y la hora en que se tomó la muestra.
    \item \texttt{metales}: Una secuencia de metales que se detectaron en la muestra.
\end{itemize}

Los invariantes de la clase aseguran que:

\begin{itemize}
    \item La ubicación no sea vacía.
    \item La secuencia de metales contenga al menos un metal.
\end{itemize}

El código en VDM++ de la clase es el siguiente:

\begin{lstlisting}
class Muestra
types
public String = seq of char;
public Date = nat1 * nat1 * nat1;
public Time = nat * nat * nat;

instance variables
private id: nat1;
public ubicacion: String;
public fecha: Date;
public hora: Time;
private metales: seq of Metal;

inv len ubicacion > 0;
inv len metales > 0;

operations
public Muestra: nat1 * String * Date * Time * seq of Metal ==> Muestra
Muestra(i, ubi, fec, hor, metal) == ( id := i; ubicacion := ubi;
	fecha := fec; hora := hor; metales := metal;
);

public getMetales: () ==> seq of Metal
getMetales() == ( return metales; );

end Muestra
\end{lstlisting}

\subsection{Clase Reporte}

La clase \textbf{Reporte} almacena la información sobre múltiples muestras y define los límites permitidos para las concentraciones de metales. Los atributos principales son:

\begin{itemize}
    \item \texttt{id}: Un identificador único para el reporte.
    \item \texttt{limiteMetales}: Un mapa que relaciona los identificadores de los metales con los límites de concentración permitidos.
    \item \texttt{muestras}: Una secuencia de muestras que se han tomado.
    \item \texttt{alerta}: Un valor booleano que indica si se ha detectado un riesgo.
\end{itemize}

Los invariantes de la clase aseguran que:

\begin{itemize}
    \item La secuencia de muestras no sea vacía.
    \item El mapa de límites de metales contenga al menos un valor.
\end{itemize}

La operación \texttt{riesgoMetal} evalúa cada muestra y verifica si la concentración de algún metal excede los límites permitidos. Si esto ocurre, se activa la alerta.

El código en VDM++ de la clase es el siguiente:

\begin{lstlisting}
class Reporte
types
public String = seq of char;

instance variables
private id: nat1;
public limiteMetales: map int to Metal;
public muestras: seq of Muestra;
public alerta: bool;

inv len muestras > 0;
inv card limiteMetales > 0;

operations
public Reporte: nat1 * map int to Metal * seq of Muestra * bool ==> Reporte
Reporte(i, lim, ms, alert) == ( id := i; limiteMetales := lim;
  muestras := ms; alerta := alert; );

public riesgoMetal: () ==> bool
riesgoMetal() == (
  for all m in set elems muestras do (
    let metales = m.getMetales() in (
      for all metal in set elems metales do (
        let limite = limiteMetales(metal.getId()) in (
          if metal.getConcentracion() > limite.getConcentracion() then
            return true;
        )
      )
    )
  );
  return false;
);

end Reporte
\end{lstlisting}
