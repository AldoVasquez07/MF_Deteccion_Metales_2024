\section{Trabajos Relacionados}
Los sistemas para detectar metales pesados en el agua pueden ser portátiles o no, lo que implica un alto consumo de recursos en algunos casos y un uso mínimo de recursos en otros. \\

La voltamperometría es un método que aplica un voltaje controlado a una muestra de agua a través de tres electrodos, según el artículo de Torres Llano \cite{torres2015}. Los átomos de mercurio de la muestra se depositan sobre el electrodo de trabajo a medida que aumenta el voltaje, generando una corriente eléctrica.\\

Esta corriente cambia según la cantidad de mercurio presente, lo que permite calcular su concentración. Este proceso es extremadamente preciso y puede detectar concentraciones de mercurio muy bajas en el agua. \\

Los autores Qi Ding, Chen Li, Haijun Wang, Chuanlai Xu y Hua Kuang \cite{ding2021} proponen el uso de técnicas electroquímicas recientes para detectar los iones de metales pesados presentes en el agua. Estos métodos permiten la detección de iones como Hg(II), Cd(II) y Pb(II) de manera rápida y efectiva. Además, se destacan por su facilidad de transporte y bajo costo.