\subsection{Definici\u00f3n de Metales Pesados}

En la industria, los metales pesados desempe\u00f1an un papel esencial debido a sus propiedades físicas y químicas \u00fanicas, como la alta densidad y su resistencia a la corrosi\u00f3n. Estos metales son ampliamente utilizados en sectores como la manufactura, la electr\u00f3nica y la minería. Sin embargo, su liberaci\u00f3n al medio ambiente, a trav\u00e9s de efluentes industriales o pr\u00e1cticas inadecuadas de disposici\u00f3n de residuos, representa un desafío significativo para la salud p\u00fablica y la sostenibilidad ambiental \cite{sall2020, zaimee2021}.\\

La toxicidad de los metales pesados, incluso en concentraciones ínfimas, ha sido documentada ampliamente. Estos elementos tienden a acumularse en los organismos vivos, causando efectos adversos acumulativos a largo plazo. Por ejemplo, el plomo y el mercurio, que son altamente persistentes en el medio ambiente, pueden bioacumularse en la cadena alimentaria, afectando tanto a la fauna como a los seres humanos \cite{matthews2022, hughes2011}.\\

De los metales pesados analizados en este trabajo, se destacan los siguientes:

\begin{enumerate}
    \item \textbf{Plomo (Pb):} Su uso en baterías, pigmentos y soldaduras lo convierte en un metal crucial en la industria. Sin embargo, la exposici\u00f3n prolongada al plomo ha sido asociada con retrasos en el desarrollo cognitivo en ni\u00f1os y disfunci\u00f3n renal en adultos \cite{matthews2022, ding2021}.
    \item \textbf{Mercurio (Hg):} Ampliamente utilizado en la minería y en dispositivos como term\u00f3metros, el mercurio puede evaporarse f\u00e1cilmente, representando un riesgo tanto para los ecosistemas acu\u00e1ticos como para la salud humana \cite{sall2020, zhumanazar2022}.
    \item \textbf{Cadmio (Cd):} Este metal, presente en baterías recargables y pinturas, se clasifica como carcinog\u00e9nico. La exposici\u00f3n prolongada puede resultar en osteoporosis y disfunci\u00f3n renal severa \cite{rahimzadeh2017, park2022}.
    \item \textbf{Ars\u00e9nico (As):} En \u00e1reas donde el agua subterr\u00e1nea est\u00e1 contaminada con ars\u00e9nico, se han observado tasas elevadas de c\u00e1ncer de piel y disfunciones cardiovasculares \cite{hughes2011, truque2024}.
    \item \textbf{Cromo (Cr):} Aunque el cromo trivalente es esencial en trazas para el metabolismo humano, el cromo hexavalente es altamente t\u00f3xico, causando irritaciones severas y aumentando el riesgo de c\u00e1ncer pulmonar \cite{shin2023, park2022}.
\end{enumerate}

\subsection{Límites de Concentraci\u00f3n y Est\u00e1ndares para Metales Pesados en el Agua}

La presencia de metales pesados en fuentes de agua potable representa una preocupaci\u00f3n prioritaria para los organismos reguladores. La Organizaci\u00f3n Mundial de la Salud (OMS) y agencias como la Agencia de Protecci\u00f3n Ambiental (EPA) de los Estados Unidos han establecido límites de concentraci\u00f3n para proteger la salud humana. Por ejemplo, el límite recomendado para el plomo es de 0.01 mg/L, mientras que para el mercurio es de 0.006 mg/L \cite{truque2024, zaimee2021}.\\

El monitoreo continuo de estas concentraciones es fundamental, ya que la exposici\u00f3n prolongada puede derivar en problemas de salud irreversibles. Estudios recientes han destacado el papel de los sensores avanzados, que utilizan materiales como nanopartículas y polímeros conductores para mejorar la sensibilidad y la especificidad en la detecci\u00f3n de metales pesados \cite{yu2022, zhumanazar2022}. Estos avances tecnol\u00f3gicos han permitido la creaci\u00f3n de dispositivos port\u00e1tiles que combinan m\u00e9todos electroquímicos y \u00f3pticos para realizar mediciones en tiempo real \cite{ding2021, park2022}.\\

El dise\u00f1o del sistema propuesto incorpora estas tecnologías emergentes para ofrecer una soluci\u00f3n eficiente y econ\u00f3mica. Adem\u00e1s, los datos obtenidos se comparan con est\u00e1ndares internacionales, permitiendo emitir alertas cuando las concentraciones exceden los valores seguros establecidos \cite{rahimzadeh2017, shin2023}. Este enfoque no solo asegura la precisi\u00f3n del sistema, sino que tambi\u00e9n facilita su implementaci\u00f3n en contextos donde los recursos son limitados.\\

\subsection{Especificaci\u00f3n Formal para Sistemas de Detecci\u00f3n}

Para garantizar la confiabilidad y robustez del sistema, se ha empleado un enfoque de especificaci\u00f3n formal. Este m\u00e9todo permite describir de manera rigurosa los comportamientos esperados del sistema, minimizando posibles ambig\u00fcedades en su implementaci\u00f3n \cite{fitzgerald2007, woodcock1996}. Herramientas como Z y otros lenguajes de especificaci\u00f3n formal han demostrado ser eficaces en el dise\u00f1o y validaci\u00f3n de sistemas críticos, como los de detecci\u00f3n de contaminantes \cite{yu2022}.\\

Adem\u00e1s, la validaci\u00f3n del sistema se realiza a trav\u00e9s de pruebas exhaustivas que simulan condiciones reales de operaci\u00f3n. Estas pruebas incluyen escenarios de detecci\u00f3n de m\u00faltiples metales en concentraciones mixtas y la evaluaci\u00f3n de la respuesta del sistema frente a interferencias químicas \cite{rahimzadeh2017, ding2021}. Este enfoque asegura que el sistema cumpla con los requisitos establecidos, brindando resultados precisos y confiables en todo momento.\\

En resumen, el uso de tecnologías avanzadas combinado con m\u00e9todos formales de especificaci\u00f3n y validaci\u00f3n constituye un enfoque integral para abordar el problema de la detecci\u00f3n de metales pesados. Este enfoque no solo mejora la calidad de los datos obtenidos, sino que tambi\u00e9n refuerza la confianza en la toma de decisiones basada en evidencia científica.
