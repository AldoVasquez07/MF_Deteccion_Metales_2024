\section{Conclusión}
El sistema de detección de metales planteado es una herramienta fundamental para el monitoreo preciso y constante de contaminantes en entornos industriales, destacando especialmente en el contexto de la contaminación del agua en nuestro país, Perú. Este sistema proporciona un control confiable sobre las concentraciones de metales, facilitando la detección oportuna de riesgos vinculados a la presencia de metales pesados. Acelerando las intervenciones y minimizando los impactos negativos tanto en el medio ambiente como en las comunidades que dependen de fuentes de agua seguras. Esto es particularmente beneficioso, debido a la gran cantidad de actividades industriales como la minería que contribuyen significativamente a la contaminación de los recursos hídricos. En este sentido, el sistema no solo aporta la capacidad de respuesta ante concentraciones peligrosas, sino que también facilita la implementación de políticas de control y reducción de la contaminación, promoviendo así la protección de la salud pública y el medio ambiente.
