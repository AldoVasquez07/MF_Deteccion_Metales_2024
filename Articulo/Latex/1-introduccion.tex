\section{Introducción}  
La contaminación del agua con metales pesados es un problema grave que afecta a muchas regiones del Perú, como Ayacucho, Junín, Piura, Puno, Moquegua y la provincia constitucional del Callao. La exposición a estos contaminantes, generados por actividades como la minería, la erosión natural, las prácticas agrícolas, el uso de fertilizantes, la mala gestión de residuos sólidos y los desechos industriales, pone en riesgo la salud y el medio ambiente.  

Metales pesados como el plomo, arsénico, mercurio y cadmio son altamente tóxicos y, al liberarse en cuerpos de agua, representan un peligro considerable para las comunidades que dependen de estos recursos hídricos. La exposición prolongada puede causar daños neurológicos, trastornos del sistema inmunológico, enfermedades renales, cáncer y afectar negativamente el desarrollo cognitivo en niños.  

A pesar de la gravedad del problema, los métodos de detección actuales suelen ser costosos, requieren equipos especializados y no permiten la detección en tiempo real, lo que incrementa el riesgo de incidentes de contaminación. En este contexto, surge la necesidad de un sistema automatizado, accesible y eficiente para la detección de metales pesados en el agua, capaz de operar en diversos contextos geográficos y económicos.  

Por ello, el presente trabajo tiene como objetivo general desarrollar un sistema basado en métodos formales para la detección automática y en tiempo real de metales pesados en el agua, contribuyendo así a la protección de la salud pública y el medio ambiente. Este sistema se enfocará en regiones vulnerables, donde el riesgo de exposición es elevado y la infraestructura actual es limitada.

\subsection{Objetivos Específicos}  
\begin{itemize}  
    \item Definir los parámetros y condiciones para el monitoreo continuo de metales pesados en el agua, como plomo, arsénico y mercurio.  
    \item Establecer los criterios para alertar a los usuarios cuando los niveles de metales pesados superen los límites permitidos.  
    \item Determinar el proceso de validación y calibración del sistema para garantizar su precisión a largo plazo.  
\end{itemize}  
